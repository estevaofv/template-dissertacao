%%%%%%%%%%%%%%%%%%%%%%%%%%%%%%%%%%%%%%%%%%%%%%%%%%%%%%%%%%%%%%%%%%%%%%%%%%%%%%%%
% CAPÍTULO 1
%%%%%%%%%%%%%%%%%%%%%%%%%%%%%%%%%%%%%%%%%%%%%%%%%%%%%%%%%%%%%%%%%%%%%%%%%%%%%%%%

\chapter{Introdução} %
\label{chap:introducao} %

De acordo com \citeonline{smid2003cnc}, o comando numérico pode ser definido
como a operação de equipamentos através de instruções específicas ao sistema de
controle da máquina. Estas instruções são combinações de letras, números e
símbolos, escritas de maneria lógica e com sintaxe predeterminada. Uma coleção
dessas instruções é chamado de ``programa CNC'' que pode ser armazenado e
reutilizado inúmeras vezes a fim de se obter uma repetibilidade de um processo
ou peça a ser produzida.

Máquinas de controle numérico computadorizado são tipicamente equipamentos
mecatrônicos diz \citeonline{suh2008theory}, ou seja, englobam máquinas
compostas por partes mecânicas e elétricas, onde o sistema de controle numérico
é um componente elétrico. Conforme \citeonline{radhakrishnan2008cad/cam/cim}, o
sistema de controle é o cérebro de uma máquina CNC e pode ser montado a fim de
efetuar o controle sobre diversas funções da máquina.

Segundo \citeonline{suh2008theory}, sistema de CNC é composto pelo módulo de
interpretação que interpreta o programa CNC, o módulo interpolador que gera o
caminho de movimento da ferramenta, o módulo de aceleração/desaceleração que
suaviza o movimento dos eixos, e por fim a unidade de controle de posicionamento
que controla os motores com base no sinal de \emph{feedback} e o resultado da
interpolação.


%%% Local Variables:
%%% mode: latex
%%% TeX-master: "../main-dissertacao"
%%% coding: utf-8
%%% End:
